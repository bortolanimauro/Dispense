\documentclass[a4paper,12pt]{article}

% Pacchetti necessari
\usepackage[utf8]{inputenc}
\usepackage[italian]{babel}
\usepackage{graphicx}
\usepackage{listings}
\usepackage{xcolor}
\usepackage{hyperref}
\usepackage{geometry}
\usepackage{fancyhdr}
\usepackage{amsmath}
\usepackage{tcolorbox}
\usepackage{enumitem}

% Impostazioni pagina
\geometry{top=2.5cm, bottom=2.5cm, left=2.5cm, right=2.5cm}

% Configurazione listings per C#
\lstdefinestyle{csharp}{
    language=[Sharp]C,
    basicstyle=\ttfamily\small,
    keywordstyle=\color{blue}\bfseries,
    commentstyle=\color{green!60!black}\itshape,
    stringstyle=\color{red},
    numbers=left,
    numberstyle=\tiny\color{gray},
    stepnumber=1,
    numbersep=10pt,
    backgroundcolor=\color{gray!10},
    frame=single,
    tabsize=4,
    breaklines=true,
    captionpos=b,
    showstringspaces=false
}

% Box personalizzati
\newtcolorbox{esercizio}{
    colback=blue!5!white,
    colframe=blue!75!black,
    title=Esercizio Pratico
}

\newtcolorbox{attenzione}{
    colback=red!5!white,
    colframe=red!75!black,
    title=ATTENZIONE!
}

\newtcolorbox{consiglio}{
    colback=green!5!white,
    colframe=green!75!black,
    title=Consiglio del Programmatore
}

% Header e footer
\pagestyle{fancy}
\fancyhf{}
\lhead{Dispensa Windows Forms}
\rhead{Programmazione C\#}
\cfoot{\thepage}

% Informazioni documento
\title{\textbf{Programmare con C\# e le Windows Forms\Animazioni e Grafica Interattiva}}
\author{Dispensa Didattica}
\date{\today}

\begin{document}

\tableofcontents
\newpage
% ====================================
% INDICE DELLA DISPENSA
% ====================================
\section*{Indice della Dispensa}
\addcontentsline{toc}{section}{Indice della Dispensa}

\subsection*{Parte 1: Le Basi della Grafica (L'Interfaccia)}
\begin{enumerate}
    \item Benvenuti nel mondo Windows Forms
    \begin{itemize}
        \item Cos'è una "finestra" (GUI) e perché si usa
        \item Come aprire Visual Studio e creare il primo progetto
    \end{itemize}
    \item Gli attrezzi del mestiere: I Controlli
    \begin{itemize}
        \item La Form: Il nostro foglio di lavoro
        \item Label, TextBox e Button
        \item La Toolbox
    \end{itemize}
    \item Far succedere le cose: Gli Eventi
    \begin{itemize}
        \item Il concetto di "Click"
        \item Il primo esercizio: "L'app che ti saluta"
    \end{itemize}
\end{enumerate}

\subsection*{Parte 2: Organizzare il Codice (Le Classi e le Liste)}
\begin{enumerate}[resume]
    \item Le Classi: Creare il "Modello" di un oggetto
    \item Le Liste: Una scatola per tanti oggetti
\end{enumerate}

\subsection*{Parte 3: Disegnare e Muovere (La Logica del Gioco)}
\begin{enumerate}[resume]
    \item Disegnare con i Pixel
    \item Il Timer: L'orologio del Programmatore
    \item Gestire la Tastiera
\end{enumerate}

\subsection*{Parte 4: Il Progetto Finale (Quadratini e Rimbalzi)}
\begin{enumerate}[resume]
    \item Creare la Classe "Quadratino"
    \item La Matematica del Rimbalzo
    \item Gli Ostacoli e le Collisioni
    \item Mettere tutto insieme
\end{enumerate}

\newpage

% ====================================
% PARTE 1
% ====================================
\part{Le Basi della Grafica (L'Interfaccia)}

\section{Benvenuti nel mondo Windows Forms}

Le Windows Forms (chiamate spesso \textbf{WinForms}) sono una tecnologia che ci permette di creare programmi per il computer con un'interfaccia grafica.

\subsection{Cosa significa "Interfaccia Grafica" (GUI)?}

Invece di avere solo una schermata nera dove scrivere comandi difficili, usiamo le \textbf{finestre}. In una finestra possiamo inserire pulsanti da cliccare, caselle dove scrivere e immagini che si muovono. È il modo in cui funzionano quasi tutti i programmi che usi ogni giorno sul PC.

\subsection{Come creare il tuo primo progetto}

Per iniziare a programmare, dobbiamo preparare il nostro "laboratorio" in Visual Studio:

\begin{enumerate}
    \item Apri \textbf{Visual Studio} sul tuo computer.
    \item Clicca su \textbf{"Crea un nuovo progetto"}.
    \item Cerca nella lista: \textbf{"App Windows Forms (.NET Framework)"} oppure \textbf{"Windows Forms App"}.
    \item Dai un nome al tuo lavoro (per esempio: \texttt{LaMiaPrimaApp}) e clicca su \textbf{Crea}.
    \item Dopo pochi istanti, vedrai apparire una finestra vuota al centro dello schermo: quella è la tua \texttt{Form1}, la base di tutto il tuo programma.
\end{enumerate}

% ====================================
\section{Gli attrezzi del mestiere: I Controlli}

Quando apriamo un progetto, Visual Studio ci mette a disposizione una \textbf{Toolbox} (la nostra Scatola degli Attrezzi). Da qui possiamo trascinare gli oggetti direttamente sulla nostra finestra (Form). Questi oggetti si chiamano \textbf{Controlli}.

\subsection{La Form: Il nostro "Foglio di Lavoro"}

La \textbf{Form} è la finestra principale dell'applicazione. È come un contenitore che ospita tutti gli altri oggetti. Possiamo cambiare il suo aspetto andando nella finestra delle \textbf{Proprietà} (di solito si trova in basso a destra):

\begin{itemize}
    \item \textbf{Text}: Cambia il titolo che appare in alto sulla finestra (es: "Il mio Gioco").
    \item \textbf{BackColor}: Serve per cambiare il colore dello sfondo (es: Bianco o Blu).
    \item \textbf{Size}: Serve per decidere quanto deve essere grande la finestra.
\end{itemize}

\subsection{I Controlli più importanti}

Ecco gli strumenti che useremo più spesso:

\begin{enumerate}
    \item \textbf{Label} (Etichetta): Serve per mostrare dei testi fissi che l'utente non può cambiare. Ad esempio: "Inserisci il tuo nome".
    \item \textbf{TextBox} (Casella di testo): È una casella bianca dove chi usa il programma può scrivere parole o numeri.
    \item \textbf{Button} (Pulsante): È il classico bottone da cliccare per far succedere qualcosa (es: un pulsante "Saluta" o "Inizia Gioco").
    \item \textbf{PictureBox} (Riquadro Immagine): Questo è fondamentale per noi! È un rettangolo speciale dove andremo a "disegnare" i nostri pixel o quadratini che si muovono.
\end{enumerate}

\subsection{Come personalizzare i controlli}

Ogni volta che aggiungi un pulsante o una casella di testo, puoi cambiare le sue \textbf{Proprietà} proprio come abbiamo fatto con la Form:

\begin{itemize}
    \item Puoi cambiare il colore di un pulsante.
    \item Puoi cambiare la scritta sopra di esso.
\end{itemize}

\begin{consiglio}
Dai sempre un nome chiaro ai tuoi controlli! Ad esempio, chiama un pulsante \texttt{btnSaluta} invece di lasciarlo come \texttt{button1}. Ti aiuterà a non confonderti quando scriverai il codice.
\end{consiglio}

\subsection{Cosa abbiamo imparato}

\begin{itemize}
    \item Le WinForms servono per creare programmi con finestre.
    \item La Form è la nostra finestra principale.
    \item Usiamo la Toolbox per trascinare pulsanti, scritte e immagini.
    \item Ogni oggetto ha delle Proprietà per cambiare colore, testo e dimensioni.
\end{itemize}

% ====================================
\subsection{Esercizio Pratico: La tua prima App "Saluta-Tutti"}

\begin{esercizio}
In questo esercizio creeremo un piccolo programma che chiede il nome all'utente e, quando si preme un pulsante, mostra un saluto personalizzato.

\textbf{Obiettivo:} Imparare a:
\begin{enumerate}
    \item Aggiungere controlli sulla finestra (Form) usando la Toolbox.
    \item Cambiare le Proprietà (come i testi e i colori).
    \item Scrivere il primo codice per far reagire il pulsante al "Click".
\end{enumerate}

\textbf{Istruzioni Passo-Passo:}
\begin{enumerate}
    \item \textbf{Crea il Progetto}: Apri Visual Studio e crea un nuovo progetto di tipo App Windows Forms.
    \item \textbf{Prepara la Finestra}:
    \begin{itemize}
        \item Clicca sulla Form (la finestra vuota).
        \item Nella finestra delle Proprietà, cerca la voce \texttt{Text} e scrivi "La mia App di Saluto".
    \end{itemize}
    \item \textbf{Aggiungi gli "Attrezzi" (Controlli)}: Trascina dalla Toolbox questi tre elementi sulla Form:
    \begin{itemize}
        \item Una \textbf{Label}: Cambia la sua proprietà \texttt{Text} in "Come ti chiami?".
        \item Una \textbf{TextBox}: Qui è dove lo studente scriverà il suo nome.
        \item Un \textbf{Button}: Cambia la sua proprietà \texttt{Text} in "Cliccami!".
        \item Una seconda \textbf{Label}: Questa servirà per il risultato. Cancella il suo testo iniziale (lasciala vuota) e chiamala nella proprietà \texttt{(Name)} come \texttt{lblRisultato}.
    \end{itemize}
\end{enumerate}
\end{esercizio}

\subsubsection{La Soluzione (Il Codice)}

Per scrivere il codice, fai \textbf{doppio clic} sul pulsante che hai creato. Visual Studio aprirà automaticamente la pagina del codice e creerà una funzione chiamata \texttt{button1\_Click}.

Inserisci questo codice tra le parentesi graffe \texttt{\{ \}}:

\lstset{style=csharp}
\begin{lstlisting}[caption={Evento Click del pulsante}]
private void button1_Click(object sender, EventArgs e)
{
    // Prendiamo il nome scritto nella TextBox e mettiamolo in una Label di saluto
    label2.Text = "Ciao, " + textBox1.Text + "! Benvenuto nel mondo C#!";
    
    // In alternativa, possiamo far apparire una finestrella pop-up
    MessageBox.Show("Hai premuto il pulsante!"); 
}
\end{lstlisting}

\subsubsection{Spiegazione semplice del codice}

\begin{itemize}
    \item \texttt{textBox1.Text}: Serve per "leggere" quello che l'utente ha scritto dentro la casella.
    \item \texttt{label2.Text = ...}: Serve per "scrivere" o cambiare il testo della nostra etichetta a schermo.
    \item \texttt{MessageBox.Show(...)}: Crea una piccola finestra di avviso che compare sopra il programma.
\end{itemize}

\subsubsection{Come verificare se funziona}

Premi il tasto \textbf{F5} (o il tasto \textbf{Start} in alto). Se tutto è corretto:
\begin{enumerate}
    \item Si aprirà la tua finestra.
    \item Scrivi il tuo nome nella casella bianca.
    \item Clicca sul pulsante.
    \item Magia! Apparirà il saluto personalizzato sulla Label che prima era vuota.
\end{enumerate}

% ====================================
\section{Far succedere le cose: Gli Eventi}

In questa parte impareremo il "cuore" della programmazione: come far reagire il computer quando noi facciamo qualcosa.

\subsection{Cos'è un Evento?}

Un \textbf{evento} è un'azione che accade mentre il programma è acceso. Immagina che il computer stia "in ascolto" e aspetti che tu faccia qualcosa. Ad esempio:

\begin{itemize}
    \item Fare un \textbf{Click} su un pulsante.
    \item Premere un tasto sulla tastiera (come le frecce per muovere un personaggio).
    \item Aprire la finestra del programma (evento \textbf{Load}).
\end{itemize}

\subsection{Il protagonista: L'evento Click}

L'evento più usato in assoluto è il \textbf{Click}. Si attiva quando l'utente preme un pulsante (Button) con il mouse.

\subsubsection{Come si crea un evento Click?}

\begin{enumerate}
    \item Vai sulla tua finestra (Form) in Visual Studio.
    \item Fai \textbf{doppio clic veloce} sul pulsante che hai creato.
    \item Magia! Visual Studio ti porterà nel codice e creerà una funzione "vuota" pronta per essere riempita.
\end{enumerate}

Esempio di codice:

\begin{lstlisting}[caption={Esempio di evento Click}]
private void button1_Click(object sender, EventArgs e)
{
    // Tutto quello che scrivi qui dentro succedera' QUANDO clicchi il pulsante!
    MessageBox.Show("Hai cliccato il pulsante!"); // Fa apparire un messaggio
}
\end{lstlisting}

\subsection{Altri eventi importanti per i nostri progetti}

Oltre al click, useremo degli eventi speciali per creare giochi o app animate:

\begin{enumerate}
    \item \textbf{Load} (Caricamento): Questo evento scatta una sola volta, proprio quando il programma si avvia. Lo useremo per preparare i nostri quadratini e decidere i loro colori iniziali.
    \item \textbf{KeyDown} (Tasto Premuto): Serve per sentire se stai premendo le frecce della tastiera. Ad esempio, possiamo dire al computer: "Se premi la freccia Su, sposta il quadratino verso l'alto".
    \item \textbf{Tick} (Il battito del Timer): È un evento speciale che si ripete da solo ogni pochi millisecondi. È fondamentale per le animazioni, perché ci permette di muovere gli oggetti continuamente senza dover cliccare nulla.
\end{enumerate}

\begin{consiglio}
Ogni volta che crei un evento, Visual Studio gli dà un nome standard (come \texttt{button1\_Click}). Per non confonderti, ricorda di dare ai tuoi pulsanti nomi chiari nella finestra delle proprietà, come \texttt{btnInizia} o \texttt{btnSaluta}. Così il tuo codice sarà ordinato e facile da leggere!
\end{consiglio}

\subsection{Mini-Esercizio: Cambia Colore!}

Prova a creare un pulsante che, ogni volta che viene cliccato, cambia il colore dello sfondo della finestra.

\textit{Aiuto: Nel codice del Click dovrai scrivere:} \texttt{this.BackColor = Color.Red;}

\subsection{Esercizio: Il Pulsante "Cambia-Colore"}

\begin{esercizio}
\textbf{Obiettivo:} Creare due pulsanti che cambiano il colore della finestra quando vengono cliccati.

\textbf{Istruzioni:}
\begin{enumerate}
    \item Trascina due pulsanti sulla Form.
    \item Cambia il testo del primo in "Rosso" e del secondo in "Verde".
    \item Fai doppio clic sul primo pulsante e scrivi la soluzione.
\end{enumerate}

\textbf{Soluzione:}
\end{esercizio}

\begin{lstlisting}[caption={Cambiare colore della Form}]
private void button1_Click(object sender, EventArgs e)
{
    this.BackColor = Color.Red; // "this" indica la finestra stessa
}

private void button2_Click(object sender, EventArgs e)
{
    this.BackColor = Color.Green; // Cambia il colore in verde
}
\end{lstlisting}

\subsection{Cosa abbiamo imparato}

\begin{itemize}
    \item Gli eventi sono le risposte del computer alle nostre azioni.
    \item Il doppio clic su un oggetto è il modo più veloce per creare un evento.
    \item Esistono eventi per il mouse (Click), per la tastiera (KeyDown) e per il tempo (Tick del Timer).
\end{itemize}

\newpage

% ====================================
% PARTE 2
% ====================================
\part{Organizzare il Codice (Le Classi e le Liste)}

\section{Le Classi: Creare il "Modello" di un oggetto}

In programmazione, una \textbf{Classe} è come una "fabbrica" o uno stampo. Serve a descrivere come deve essere fatto un oggetto e cosa può fare.

\begin{itemize}
    \item \textbf{Le Proprietà}: Sono le caratteristiche (es: il nome, il colore, la posizione).
    \item \textbf{I Metodi}: Sono le azioni che l'oggetto sa compiere (es: camminare, salutare).
    \item \textbf{Il Costruttore}: È un comando speciale che serve a creare l'oggetto "nuovo di zecca" assegnandogli subito i suoi valori iniziali.
\end{itemize}

\textbf{Esempio reale:} La classe "Automobile" descrive che ogni auto ha un colore e una velocità. La tua macchina specifica è l'oggetto creato da quella classe.

\subsection{Esercizio: Crea la tua Classe "Studente"}

\begin{esercizio}
\textbf{Obiettivo:} Creare una classe che rappresenti uno studente con un nome e un voto.

\textbf{Istruzioni:}
\begin{enumerate}
    \item Vai nel codice del tuo progetto.
    \item Fuori dalla zona della Form, scrivi la classe Studente.
    \item Crea un pulsante nella Form che, quando cliccato, crea uno studente e mostra i suoi dati.
\end{enumerate}
\end{esercizio}

\textbf{Soluzione:}

\begin{lstlisting}[caption={Classe Studente}]
// 1. Definiamo la "ricetta" dello studente
class Studente
{
    public string Nome;
    public int Voto;

    // Metodo per mostrare le informazioni
    public string Scheda()
    {
        return Nome + " - Voto: " + Voto;
    }
}

// 2. Codice dentro il pulsante per usare la classe
private void btnCrea_Click(object sender, EventArgs e)
{
    Studente s1 = new Studente(); // Creiamo l'oggetto
    s1.Nome = "Mario";
    s1.Voto = 8;
    MessageBox.Show(s1.Scheda()); // Mostra: Mario - Voto: 8
}
\end{lstlisting}

% ====================================
\section{Le Liste: Una scatola per tanti oggetti}

A volte non ci basta un solo oggetto, ma ne servono tanti (pensa a una lista della spesa o a tanti nemici in un gioco). Per questo usiamo la \textbf{List}.

A differenza di un contenitore rigido, la lista è \textbf{dinamica}: puoi aggiungere o togliere elementi quando vuoi.

\subsection{Comandi principali da ricordare}

\begin{itemize}
    \item \textbf{Add}: Per aggiungere qualcuno alla lista.
    \item \textbf{Remove}: Per togliere qualcuno.
    \item \textbf{Count}: Per sapere quanti elementi ci sono in totale.
    \item \textbf{Indice [i]}: Ricorda che il computer inizia a contare da 0! Il primo della lista è sempre lo studente numero 0.
\end{itemize}

\subsection{Esercizio: La Lista dei Nomi}

\begin{esercizio}
\textbf{Obiettivo:} Creare un'app dove scrivi un nome in una TextBox, premi un pulsante e il nome finisce in una ListBox (un elenco visibile).

\textbf{Istruzioni:}
\begin{enumerate}
    \item Aggiungi una TextBox, un Button e una ListBox sulla Form.
    \item Crea una lista di parole (stringhe) all'inizio del codice.
\end{enumerate}
\end{esercizio}

\textbf{Soluzione:}

\begin{lstlisting}[caption={Gestione di una lista}]
// Creiamo la nostra "scatola" vuota
List<string> nomi = new List<string>();

private void btnAggiungi_Click(object sender, EventArgs e)
{
    nomi.Add(textBox1.Text); // Aggiungiamo il nome scritto nella casella
    
    // Aggiorniamo la ListBox per far vedere i nomi a schermo
    listBox1.DataSource = null;              
    listBox1.DataSource = nomi;  
}
\end{lstlisting}

\subsection{Cosa abbiamo imparato in questa parte}

\begin{itemize}
    \item La \textbf{Classe} è il modello (lo stampo), l'\textbf{Oggetto} è ciò che creiamo.
    \item Le \textbf{Liste} servono a gestire tanti oggetti insieme in modo ordinato.
    \item Possiamo collegare le liste ai componenti grafici come la ListBox per vedere i nostri dati.
\end{itemize}

\newpage

% ====================================
% PARTE 3
% ====================================
\part{Disegnare e Muovere (La Logica del Gioco)}

\section{Disegnare con i Pixel}

Per creare disegni o giochi, non usiamo solo pulsanti pronti, ma impariamo a colorare lo schermo punto per punto. Questi punti si chiamano \textbf{Pixel}.

\subsection{Gli strumenti per disegnare}

Per disegnare in C\# ci servono tre cose fondamentali:

\begin{enumerate}
    \item \textbf{PictureBox}: È la cornice o il "quadro" che mettiamo sulla nostra Form per far vedere i disegni.
    \item \textbf{Bitmap}: È il nostro "foglio digitale". Immaginalo come un foglio a quadretti trasparente grande quanto la PictureBox.
    \item \textbf{SetPixel}: È il comando che usiamo per colorare un singolo quadretto (pixel) in una posizione precisa (x, y) con un colore scelto.
\end{enumerate}

Esempio di codice:

\begin{lstlisting}[caption={Disegnare un pixel}]
// Creiamo un foglio digitale grande quanto la nostra PictureBox
Bitmap bmp = new Bitmap(pictureBox1.Width, pictureBox1.Height);

// Coloriamo un pixel di rosso nelle coordinate x=10 e y=20
bmp.SetPixel(10, 20, Color.Red);

// Diciamo alla PictureBox di mostrare il foglio che abbiamo disegnato
pictureBox1.Image = bmp;
\end{lstlisting}

\subsection{Esercizio: Disegna un quadratino rosso}

\begin{esercizio}
\textbf{Obiettivo:} Colorare un piccolo quadrato di 10x10 pixel al centro dello schermo.
\end{esercizio}

\textbf{Soluzione:}

\begin{lstlisting}[caption={Disegnare un quadrato}]
Bitmap foglio = new Bitmap(pictureBox1.Width, pictureBox1.Height);
for (int i = 50; i < 60; i++) // Ciclo per la larghezza
{
    for (int j = 50; j < 60; j++) // Ciclo per l'altezza
    {
        foglio.SetPixel(i, j, Color.Red);
    }
}
pictureBox1.Image = foglio;
\end{lstlisting}

% ====================================
\section{Il Timer: L'orologio del Programmatore}

Per far muovere qualcosa, dobbiamo ridisegnarlo tante volte in posizioni leggermente diverse, proprio come nei cartoni animati. Il \textbf{Timer} è lo strumento che dice al computer quando è il momento di fare un passo avanti.

\subsection{Come funziona il Timer?}

\begin{itemize}
    \item \textbf{Interval}: È la velocità del "battito". Si misura in millisecondi (ms). Se impostiamo 20, il timer farà un "tic" 50 volte al secondo.
    \item \textbf{Enabled}: Serve per accendere (\texttt{true}) o spegnere (\texttt{false}) il movimento.
    \item \textbf{Evento Tick}: È la funzione che viene eseguita a ogni "tic" dell'orologio.
\end{itemize}

\subsection{La logica del movimento}

A ogni "tic" del Timer dobbiamo:

\begin{enumerate}
    \item \textbf{Cancellare} l'oggetto dalla vecchia posizione (colorandolo di bianco).
    \item \textbf{Aggiornare} la posizione (es. aggiungere 1 alla coordinata X).
    \item \textbf{Ridisegnare} l'oggetto nella nuova posizione.
\end{enumerate}

\subsection{Esercizio: Il Punto che cammina}

\begin{esercizio}
\textbf{Obiettivo:} Far muovere un punto orizzontalmente da solo.
\end{esercizio}

\textbf{Soluzione:}

\begin{lstlisting}[caption={Punto in movimento}]
int x = 10; // Posizione di partenza

private void timer1_Tick(object sender, EventArgs e)
{
    x = x + 2; // Spostiamo il punto di 2 pixel a ogni tic
    disegnaPunto(x, 50); // Funzione che colora il pixel
}
\end{lstlisting}

\begin{attenzione}
\textbf{L'Errore più Comune!}

Guardate bene dove abbiamo dichiarato la variabile \texttt{int x = 10;}. L'abbiamo scritta \textbf{fuori} dalla funzione \texttt{timer1\_Tick}, all'inizio della classe Form.

\textbf{Perché?}

\begin{itemize}
    \item \textbf{Variabile FUORI (Globale/di Classe)}: La variabile nasce quando parte il programma e mantiene la sua memoria per tutto il tempo. Ogni volta che il Timer scatta, ritrova il valore precedente modificato.
    \item \textbf{Variabile DENTRO (Locale)}: Se scriveste \texttt{int x = 10;} dentro le parentesi graffe del \texttt{timer1\_Tick}, la variabile verrebbe creata, impostata a 10, aumentata a 15 e poi distrutta subito dopo. Al prossimo scatto del Timer? Verrebbe creata di nuovo a 10. Il vostro oggetto sembrerebbe immobile!
\end{itemize}

\textbf{Regola d'oro:} Se un dato deve cambiare nel tempo e ricordarsi il valore precedente (come la posizione di un personaggio), deve essere dichiarato fuori dai metodi.
\end{attenzione}

% ====================================
\section{Gestire la Tastiera: Comandiamo noi!}

Vogliamo che il nostro quadratino si muova quando premiamo le frecce della tastiera? Per farlo usiamo l'evento \textbf{KeyDown}.

\subsection{Configurazione Iniziale}

Per far sì che la finestra "senta" i tasti, dobbiamo attivare una proprietà speciale della Form chiamata \texttt{KeyPreview = true;}.

\subsection{Riconoscere i tasti}

Usiamo un comando chiamato \texttt{switch} che serve a scegliere cosa fare in base al tasto premuto:

\begin{itemize}
    \item \texttt{Keys.Up}: Freccia Su (diminuiamo la Y).
    \item \texttt{Keys.Down}: Freccia Giù (aumentiamo la Y).
    \item \texttt{Keys.Left}: Freccia Sinistra (diminuiamo la X).
    \item \texttt{Keys.Right}: Freccia Destra (aumentiamo la X).
\end{itemize}

\subsection{Esercizio: Pilota il tuo pixel}

\begin{esercizio}
\textbf{Obiettivo:} Muovere un pixel sullo schermo usando le frecce della tastiera.
\end{esercizio}

\textbf{Soluzione:}

\begin{lstlisting}[caption={Controllo da tastiera}]
private void Form1_KeyDown(object sender, KeyEventArgs e)
{
    switch(e.KeyCode)
    {
        case Keys.Up:    y -= 5; break; // Va su
        case Keys.Down:  y += 5; break; // Va giu
        case Keys.Left:  x -= 5; break; // Va a sinistra
        case Keys.Right: x += 5; break; // Va a destra
    }
    
    // Controlliamo che non esca dallo schermo
    if (x < 0) x = 0;
    if (y < 0) y = 0;
    
    DisegnaPunto(); // Ridisegna nella nuova posizione
}
\end{lstlisting}

\subsection{Cosa abbiamo imparato in questa parte}

\begin{itemize}
    \item Usiamo la \textbf{Bitmap} come un foglio e \textbf{SetPixel} come un pennello.
    \item Il \textbf{Timer} crea le animazioni ripetendo azioni a intervalli regolari.
    \item Con \textbf{KeyDown} possiamo dare comandi al programma usando la tastiera.
\end{itemize}

\newpage

% ====================================
% PARTE 4
% ====================================
\part{Il Progetto Finale (Quadratini e Rimbalzi)}

\section{Prima di calcolare i rimbalzi: Com'è fatto il nostro Quadratino?}

Per gestire le collisioni nel prossimo capitolo, useremo delle proprietà specifiche. Assicuriamoci che la vostra classe \texttt{Quadratino} abbia tutte le informazioni necessarie: posizione (x, y), velocità (vx, vy) e grandezza (dimensione).

Ecco come deve apparire il codice della vostra classe per funzionare con gli esempi successivi:

\begin{lstlisting}[caption={Classe Quadratino completa}]
public class Quadratino
{
    // 1. PROPRIETA' (Le caratteristiche)
    public int x;          // Posizione orizzontale
    public int y;          // Posizione verticale
    public int vx;         // Velocita' orizzontale (passo)
    public int vy;         // Velocita' verticale (passo)
    public int dimensione; // Grandezza del lato
    public Color colore;   // Colore del quadrato

    // 2. COSTRUTTORE (Come nasce l'oggetto)
    // Ci permette di creare quadratini diversi tra loro
    public Quadratino(int startX, int startY, int velX, int velY, 
                      int dim, Color c)
    {
        x = startX;
        y = startY;
        vx = velX;
        vy = velY;
        dimensione = dim;
        colore = c;
    }
}
\end{lstlisting}

\textit{Ora che abbiamo definito \texttt{x}, \texttt{y} e \texttt{dimensione}, possiamo usare la matematica per capire quando toccano i bordi!}

% ====================================
\section{Il ricalcolo della posizione}

Il ricalcolo della posizione di un oggetto a ogni "battito" del timer segue una logica precisa che permette di creare l'illusione del movimento. Immagina che il Timer sia un orologio che, a ogni tick, dà il via a una serie di azioni.

\subsection{La sequenza delle operazioni}

All'interno dell'evento \texttt{timer1\_Tick}, il programma esegue questi quattro compiti in ordine:

\begin{enumerate}
    \item \textbf{Cancella}: Il quadratino viene cancellato dalla sua posizione attuale (colorando i pixel di bianco).
    \item \textbf{Sposta}: Viene calcolata la nuova posizione matematica.
    \item \textbf{Controlla}: Il computer verifica se il quadratino ha toccato i bordi della finestra o un ostacolo.
    \item \textbf{Disegna}: Il quadratino viene colorato nella nuova posizione.
\end{enumerate}

\subsection{La matematica del movimento (Sposta)}

Per muovere l'oggetto, sommiamo la sua velocità alla sua posizione attuale:

\begin{itemize}
    \item Nuova X = Posizione X precedente + Velocità Orizzontale (vx).
    \item Nuova Y = Posizione Y precedente + Velocità Verticale (vy).
\end{itemize}

Nel codice, questo si scrive in modo molto semplice:

\begin{lstlisting}
q.x = q.x + q.vx; 
q.y = q.y + q.vy;
\end{lstlisting}

\subsection{Gestire i rimbalzi (Controlla)}

Dopo aver calcolato la nuova posizione, il programma deve assicurarsi che l'oggetto non esca dallo schermo. Se il quadratino tocca un bordo, la sua velocità viene invertita:

\begin{itemize}
    \item \textbf{Rimbalzo a destra o sinistra}: Se \texttt{q.x} supera i limiti della larghezza, la velocità \texttt{vx} cambia segno (\texttt{q.vx = -q.vx}).
    \item \textbf{Rimbalzo in alto o in basso}: Se \texttt{q.y} supera i limiti dell'altezza, la velocità \texttt{vy} cambia segno (\texttt{q.vy = -q.vy}).
\end{itemize}

Inoltre, viene chiamata la funzione \texttt{controllaCollisioneOstacoli(q, o1)} per vedere se il quadratino ha colpito un muro o un blocco grigio all'interno della finestra.

\subsection{Esempio di codice riassuntivo}

\begin{lstlisting}[caption={Funzione di ricalcolo posizione}]
public Quadratino ricalcolaPosizione(Quadratino q)
{
    // 1. Muovi l'oggetto
    q.x = q.x + q.vx;
    q.y = q.y + q.vy;

    // 2. Controlla se sbatte contro i bordi e inverti la marcia
    if (q.x - q.dimensione / 2 <= 0 || 
        q.x + q.dimensione / 2 >= pictureBox1.Width)
    {
        q.vx = -q.vx; // Inverte direzione orizzontale
    }

    if (q.y - q.dimensione / 2 <= 0 || 
        q.y + q.dimensione / 2 >= pictureBox1.Height)
    {
        q.vy = -q.vy; // Inverte direzione verticale
    }

    // 3. Controlla se sbatte contro l'ostacolo
    controllaCollisioneOstacoli(q, o1);

    return q;
}
\end{lstlisting}

% ====================================
\section{Gli Ostacoli e le Collisioni}

Per far rimbalzare un quadratino contro un ostacolo, dobbiamo usare la logica del "confronto dei bordi". Immagina che il quadratino e l'ostacolo siano due scatole: dobbiamo capire se queste scatole si stanno sovrapponendo.

\subsection{Creiamo l'Ostacolo}

Prima di tutto, definiamo cos'è un ostacolo. Un ostacolo è un rettangolo che ha due punti fissi: l'angolo in alto a sinistra (x0, y0) e l'angolo in basso a destra (x1, y1).

\subsection{La logica del "Tocco" (Collisione)}

Per capire se il quadratino colpisce l'ostacolo, il computer controlla due cose:

\begin{enumerate}
    \item \textbf{Controllo Orizzontale}: Il quadratino si trova alla stessa "altezza" dell'ostacolo? Se sì, sta toccando il bordo destro o sinistro?
    \item \textbf{Controllo Verticale}: Il quadratino si trova nella stessa "corsia" dell'ostacolo? Se sì, sta toccando il bordo superiore o inferiore?
\end{enumerate}

\subsection{Cosa succede quando sbatte?}

Quando il computer capisce che c'è un tocco, dobbiamo fare due azioni immediate:

\begin{itemize}
    \item \textbf{Sposta}: Spostiamo il quadratino appena fuori dall'ostacolo (per evitare che rimanga "incastrato" dentro).
    \item \textbf{Inverti}: Cambiamo la direzione della velocità. Se stava andando a destra (vx positivo), lo facciamo andare a sinistra (vx diventa negativo).
\end{itemize}

\subsection{Il Codice della Collisione (Semplificato)}

Ecco come appare la funzione che controlla se il quadratino \texttt{q} sbatte contro l'ostacolo \texttt{o}:

\begin{lstlisting}[caption={Controllo collisioni con ostacoli}]
public void controllaCollisioneOstacoli(Quadratino q, Ostacolo o)
{
    // 1. Controlliamo se il quadratino e' all'altezza giusta 
    //    per colpire i lati
    if (q.y + q.dimensione / 2 >= o.y0 && 
        q.y - q.dimensione / 2 <= o.y1)
    {
        // Se sbatte contro la faccia sinistra dell'ostacolo 
        // mentre va a destra
        if (q.x + q.dimensione / 2 >= o.x0 && q.vx > 0 && 
            q.x + q.dimensione / 2 <= o.x0 + q.vx)
        {
            q.vx = -q.vx; // Rimbalza!
        }
        
        // Se sbatte contro la faccia destra dell'ostacolo 
        // mentre va a sinistra
        if (q.x - q.dimensione / 2 <= o.x1 && q.vx < 0 && 
            q.x - q.dimensione / 2 >= o.x1 + q.vx)
        {
            q.vx = -q.vx; // Rimbalza!
        }
    }
}
\end{lstlisting}

% ====================================
\section{Mettere tutto insieme: Il Ciclo di Movimento}

Ora che sappiamo come gestire i rimbalzi, dobbiamo inserire tutto nel Timer, che è il motore del nostro programma.

Ogni volta che il timer fa "Tick", il programma segue questi 4 passi:

\begin{enumerate}
    \item \textbf{Cancella}: Colora di bianco la vecchia posizione del quadratino.
    \item \textbf{Sposta}: Calcola la nuova posizione aggiungendo la velocità (es: \texttt{x = x + vx}).
    \item \textbf{Controlla}: Verifica se il quadratino ha toccato i bordi della finestra o l'ostacolo. Se sì, cambia la velocità.
    \item \textbf{Disegna}: Colora il quadratino nella sua nuova posizione.
\end{enumerate}

\subsection{Esercizio: Il Muro Grigio}

\begin{esercizio}
\textbf{Obiettivo:} Aggiungi un rettangolo grigio al centro dello schermo e fai in modo che il tuo quadratino rosso ci rimbalzi contro.

\textbf{Istruzioni:}
\begin{enumerate}
    \item Crea un oggetto \texttt{Ostacolo} chiamato \texttt{o1} nel \texttt{Load} della Form e dagli delle coordinate (es: x0=200, y0=200, x1=300, y1=300).
    \item Nel \texttt{timer1\_Tick}, dopo aver aggiornato la posizione del quadratino, chiama la funzione \texttt{controllaCollisioneOstacoli(q1, o1)}.
\end{enumerate}
\end{esercizio}

\textbf{Soluzione (Logica):}

\begin{lstlisting}[caption={Ciclo completo nel Timer}]
// Dentro il Timer
cancellaQuadrato(q1);                // Cancella vecchio
q1.x += q1.vx;                       // Muovi
q1.y += q1.vy;
controllaCollisioneOstacoli(q1, o1); // Controlla se sbatte contro il muro
disegnaQuadrato(q1);                 // Disegna nuovo
\end{lstlisting}

\subsection{Cosa abbiamo imparato}

\begin{itemize}
    \item Le collisioni si gestiscono confrontando le coordinate dei bordi (X e Y).
    \item Per far rimbalzare un oggetto, basta moltiplicare la sua velocità per -1 (es: \texttt{vx = -vx}).
    \item Il Timer coordina tutto: cancella, muove, controlla e ridisegna.
\end{itemize}

% ====================================
\section{Gestire più quadratini contemporaneamente}

Gestire più quadratini contemporaneamente è un passaggio fondamentale per rendere le tue applicazioni più dinamiche e interessanti. Per farlo, utilizziamo la programmazione a oggetti, che ci permette di creare molteplici copie (istanze) partendo dallo stesso stampo (classe).

\subsection{Creare le "Istanze" dei Quadratini}

Invece di creare un solo oggetto, dichiariamo diversi quadratini all'interno della classe della nostra finestra (Form):

\begin{lstlisting}
Quadratino q1 = new Quadratino();
Quadratino q2 = new Quadratino();
Quadratino q3 = new Quadratino();
\end{lstlisting}

\subsection{Dare caratteristiche diverse a ognuno}

Perché i quadratini non si sovrappongano e si muovano in modo unico, dobbiamo assegnare a ciascuno valori iniziali differenti nell'evento \texttt{Load} della Form:

\begin{itemize}
    \item \textbf{Posizione}: \texttt{q1} potrebbe partire in alto a sinistra (100, 100), mentre \texttt{q2} al centro (300, 200).
    \item \textbf{Velocità}: Assegnando velocità diverse (es. \texttt{q1.vx = 2} e \texttt{q2.vx = -1}), i quadratini si muoveranno a ritmi differenti e in direzioni opposte.
    \item \textbf{Aspetto}: Puoi cambiare la dimensione e il colore per distinguerli facilmente (es. uno rosso, uno verde e uno blu).
\end{itemize}

\subsection{Gestire il movimento nel Timer}

Il segreto per farli muovere tutti insieme "nello stesso momento" è scrivere le istruzioni per ogni quadratino all'interno dell'evento \texttt{Tick} del Timer.

Il computer è così veloce che eseguirà questi comandi in sequenza per ogni oggetto, dandoci l'illusione che si muovano contemporaneamente:

\begin{enumerate}
    \item \textbf{Cancella tutti}: \texttt{cancellaQuadrato(q1);}, \texttt{cancellaQuadrato(q2);}, ecc.
    \item \textbf{Sposta tutti}: Ricalcoli la posizione per ognuno (\texttt{q1 = ricalcolaPosizione(q1);}, ecc.).
    \item \textbf{Disegna tutti}: Li mostri nella nuova posizione (\texttt{disegnaQuadrato(q1);}, ecc.).
\end{enumerate}

\subsection{Il trucco degli esperti: Usare le Liste}

Se i quadratini diventano tanti (ad esempio 10 o 100), invece di scrivere i nomi uno per uno, usiamo una Lista (\texttt{List<Quadratino>}).

Con una lista, puoi usare un ciclo \texttt{foreach} per dire al computer: "Per ogni quadratino che trovi nella mia lista, cancellalo, muovilo e ridisegnalo". Questo rende il codice molto più corto e ordinato.

\subsection{Esercizio: "La Danza dei Quadratini"}

\begin{esercizio}
\textbf{Obiettivo:} Gestire tre quadratini che rimbalzano con colori e velocità diverse.
\end{esercizio}

\textbf{Soluzione (Logica nel Timer):}

\begin{lstlisting}[caption={Gestione multipla nel Timer}]
private void timer1_Tick(object sender, EventArgs e)
{
    // 1. Fase di cancellazione (pulizia vecchia posizione)
    cancellaQuadrato(q1);
    cancellaQuadrato(q2);
    cancellaQuadrato(q3);

    // 2. Fase di ricalcolo (aggiornamento velocita' e rimbalzi)
    q1 = ricalcolaPosizione(q1);
    q2 = ricalcolaPosizione(q2);
    q3 = ricalcolaPosizione(q3);

    // 3. Fase di disegno (mostra nuova posizione)
    disegnaQuadrato(q1);
    disegnaQuadrato(q2);
    disegnaQuadrato(q3);
}
\end{lstlisting}

\subsection{Cosa abbiamo imparato}

\begin{itemize}
    \item Ogni quadratino è un oggetto indipendente con le sue proprietà (x, y, velocità, colore).
    \item Il Timer gestisce tutti i quadratini in sequenza a ogni suo "battito".
    \item Le Liste sono lo strumento migliore per gestire grandi quantità di oggetti senza fatica.
\end{itemize}

% ====================================
% CONCLUSIONE
% ====================================
\newpage
\section*{Conclusioni}
\addcontentsline{toc}{section}{Conclusioni}

In questa dispensa abbiamo imparato:

\begin{enumerate}
    \item Come creare interfacce grafiche con Windows Forms.
    \item Come gestire eventi (Click, KeyDown, Timer Tick).
    \item Come organizzare il codice con Classi e Liste.
    \item Come disegnare pixel per pixel usando Bitmap.
    \item Come creare animazioni fluide con il Timer.
    \item Come gestire collisioni e rimbalzi.
    \item Come coordinare multipli oggetti in movimento.
\end{enumerate}

\vspace{1cm}

\begin{center}
\textit{Ora sei pronto per creare i tuoi primi giochi e applicazioni interattive!}

\vspace{0.5cm}
\textbf{Buona programmazione!}
\end{center}

\end{document}
