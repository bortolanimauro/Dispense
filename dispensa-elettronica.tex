\documentclass{article}
\usepackage{graphicx}
\usepackage{color}
\usepackage{fancyhdr}
\setlength{\textheight}{8.5in}
\setlength{\textwidth}{6.5in}
\setlength{\oddsidemargin}{0in}
\setlength{\topmargin}{-0.5in}
\renewcommand{\headrulewidth}{0pt}

\definecolor{exercisecolor}{rgb}{0.8, 0.9, 0.8}
\definecolor{warningcolor}{rgb}{1, 0.8, 0.8}
\definecolor{tipcolor}{rgb}{1, 1, 0.8}

\newcommand{\exercisebox}[1]{\begin{center}\color{exercisecolor} \fbox{\parbox{6.5in}{\textbf{Esercizio:} #1}}\end{center}}
\newcommand{\warningbox}[1]{\begin{center}\color{warningcolor} \fbox{\parbox{6.5in}{\textbf{Attenzione:} #1}}\end{center}}
\newcommand{\tipbox}[1]{\begin{center}\color{tipcolor} \fbox{\parbox{6.5in}{\textbf{Consiglio:} #1}}\end{center}}

\title{Dispensa Elettronica}
\author{Bortolanimauro}
\date{2026-02-16}

\begin{document}
\maketitle
\tableofcontents

\section{Introduzione}
Questa dispensa è stata progettata per fornire una guida all'utilizzo delle interfacce grafiche in C#. Segue un approccio passo-passo e assicura che ogni concetto sia trattato in dettaglio.

\section{Struttura del progetto}
Per iniziare, è importante comprendere la struttura di un progetto Windows Forms. Di seguito, sono riportate le cartelle e i file principali...

\exercisebox{Crea un nuovo progetto Windows Forms nelle tue IDE preferite.}

\section{Stile e design}
Il design della tua applicazione deve essere coerente. Assicurati di utilizzare un font leggibile, colori armoniosi e una disposizione ordinata degli elementi dell'interfaccia utente. I box forniranno chiarezza visiva e miglioreranno l'usabilità.

\tipbox{Usa \texttt{Segoe UI} come font principale per garantire leggibilità.}

\section{Conclusioni}
Seguendo questa dispensa, sarai in grado di creare applicazioni Windows Forms efficaci e user-friendly. Ricorda sempre di testare e chiedere feedback per migliorare la tua applicazione.
\end{document}